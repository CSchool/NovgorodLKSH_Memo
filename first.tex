\section{Первая программа на С++}
\subsection{Минимальная программа}

Пример минимальной программы:

\lstset{style=CPlusPlus}
\begin{lstlisting}
int main()
{
    // nothing interesting in this program...

    /*
        Seriously, we doesn't have even message about Hello World!
        I hope that we fix this in another section...
    */
    return 0;
}
\end{lstlisting}

В данном примере представлена функций \lstinline|main|, которая является точкой входа программы. Может быть сколько угодно функций в программе, но функция \lstinline|main| обязана присутствовать среди них, причем в единственном экземпляре! Исполнение всей программы начинается именно с этой функции. \lstinline|int| перед названием функции говорит о том, что данная функция должна вернуть целочисленное значение. \lstinline|main| возвращает результат выполнения программы~---~\lstinline|return 0| означает, что программа выполнилась без ошибки и корректно завершила свою работу. В противном случае необходимо вернуть ненулевое значение.

Обратите внимание, что при помощи $//$ можно комментировать программу~---~то есть на данной строке ничего выполняться не будет. Комментарий можно растянуть на несколько линий, обрамив его между $/*$ и $*/$. Примеры комментариев представлены в коде сверху.

\subsection{Программа <<Hello, world!>>, запись в консоль данных}

Усложним программу, добавив вывод строки <<Hello, world!>> на экран консольного приложения:

\begin{lstlisting}
#include <iostream>

int main()
{
    std::cout << "Hello, world!" << std::endl;
    return 0;
}
\end{lstlisting}

В данной программе, по отношению к предыдущей, добавились две строчки:

\begin{itemize}
    \item \lstinline!#include <iostream>!
    \item \lstinline|std::cout << "Hello, world!" << std::endl;|
\end{itemize}

\lstinline!#include <iostream>! является директивой (командой) препроцессора. Данные строки обрабатываются им \emph{до} компиляции программы. Данная команда позволяет препроцессору включить в программу содержимое заголовочного файла \lstinline!iostream!, который отвечает за работу потоков ввода\textbackslash вывода. Данный файл необходимо включать в любую программу, где ожидается работа с консолью~---~когда данные считываются с консоли и когда необходимо передать какие-нибудь данные консоли.

Вся строка \lstinline|std::cout << "Hello, world!" << std::endl;| является оператором. Каждый оператор в языке оканчивается символом $;$.

Ввод\textbackslash вывод в С++ осуществляется с помощью символьных потоков:
\begin{itemize}
    \item \lstinline!std::cin!~---~стандартный входной поток;
    \item \lstinline!std::cout!~---~стандартный выходной поток;
    \item \lstinline!std::cerr!~---~стандартный поток ошибок.
\end{itemize}

Обратите внимание, что перед \lstinline|cin|, \lstinline|cout| и \lstinline|cerr| помещено \texttt{std::}. Это означает, что данные потоки принадлежат \textit{пространству имен} (\textit{namespace}) \texttt{std}. Пространства имен нужны для удобной организации объектов и операторов по смыслу.

Операция \lstinline|<<| называется \emph{операцией передачи в поток}. При выполнении этой операции то, что стоит справа, передается в поток, который стоит слева. Обратите внимание на направленность данной операции~---~данные передаются в поток справа налево! \lstinline|std::endl| является функцией, которая возвращает символ перевода на новую строчку.

Для того, чтобы постоянно не использовать в коде определенное пространство имен, можно воспользоваться двумя методами:
\begin{enumerate}
    \item Использовать для текущей файла всё пространство имен целиком:
    \begin{lstlisting}
#include <iostream>
using namespace std;

int main()
{
    cout << "Hello, world!" << endl;
    return 0;
}
    \end{lstlisting}

    \item Вынести отдельные объекты и операции:
    \begin{lstlisting}
#include <iostream>
using std::cout;
using std::endl;

int main()
{
    cout << "Hello, world!" << endl;
    return 0;
}
    \end{lstlisting}
\end{enumerate}

Рекомендуется использовать второй метод, если количество объявлений не очень большое, в противном случае не возбраняется подключать все пространство имен.

Записывать в стандартный выходной поток можно не только строки, но и данные других типов:

\begin{lstlisting}
#include <iostream>
int main()
{
    int a = 5;
    int b = 7;
    std::cout << a + b << std::endl;
}
\end{lstlisting}

\subsection{Чтение данных с консоли}

С помощью потока \lstinline|std::cin| можно считывать данные, которые вводятся с консоли:
\begin{lstlisting}
#include <iostream>
int main()
{
    int a = 0;
    int b = 0;
    int c = 0;

    std::cout << "Enter a:" << std::endl;
    std::cin >> a;

    std::cout << "Enter b:" << std::endl;
    std::cin >> b;

    std::cout << "a + b = " << a + b << std::endl;
}
\end{lstlisting}

Операция $<<$ называется \textit{операцией извлечения из потока}. При выполнении данной операции данные из стандартного входного потока слева будут переданы переменной, которая находится справа.

\section{Справочный материал по С++}

\subsection{Переменные (объявление, типы, операции с переменными)}

\subsubsection{Объявление переменных}
Синтаксис объявления переменных:

\lstinline|<Modification> <Type> variableName = variableValue;|

\lstinline|Type| --- тип переменной, \lstinline|Modification| --- изменение (модификация), накладываемое на переменную, \lstinline|variableName| --- имя переменной, \lstinline|variableValue| --- значение переменной.

После объявления переменной имеет смысл назначить ей какое-либо допустимое значение. Даже если переменная будет изменяться в дальнейшем коде программы, правилом хорошего тона является выставления значения по умолчанию, в противном случае при выполнении программы в этих переменных может оказаться <<мусор>> (случайные значения).

Инициализацию переменной можно двумя способами:
\begin{itemize}
    \item Через оператор \lstinline{=} :

        \lstinline|int a = 5;|
    \item Через оператор \lstinline{()} --- инициализация переменной с помощью конструктора:

         \lstinline|int a(5);|
\end{itemize}

На одной строке можно объявлять несколько переменных одного типа:
\begin{lstlisting}
int p = 2,k = 3;
double s,m; // bad thing, because there is no initialization!
\end{lstlisting}

\subsubsection{Типы переменных}
Стандартные типы:
\begin{enumerate}
    \item \lstinline|bool| --- логический тип;
    \item \lstinline|char| --- целочисленный тип, используется для представления символов ('a', 'z', '.', \ldots);
    \item \lstinline|int| --- целочисленный тип, используется для хранения \textbf{целых} чисел;
    \item \lstinline|float| --- вещественный тип, используется для хранения \textbf{дробных} чисел;
    \item \lstinline|double| --- вещественный тип, используется для хранения \textbf{дробных} чисел, обладает большей точностью, чем тип \lstinline|float|.
\end{enumerate}

Возможные модификаторы:
\begin{enumerate}
    \item \lstinline|const| --- запрет менять значение переменной в ходе программы. Если попытаться поменять где-нибудь константу, то компилятор сообщит о ошибке:
    \begin{lstlisting}
const double pi = 3.1415;
pi = 3.0; // This is wrong!!!
    \end{lstlisting}
    \item \lstinline|unsigned| --- переменная будет принимать только положительные значения;
    \item \lstinline|short|--- укорачивает диапазон переменных;
    \item \lstinline|long| --- увеличивает диапазон переменных.
\end{enumerate}

Размерности переменных:
\begin{table}[ht]
    \centering
    \begin{tabular}{|>{\centering\arraybackslash}m{4cm}|>{\centering\arraybackslash}m{2cm}|>{\centering\arraybackslash}m{10cm}|}
        \hline
        \textit{Тип} & \textit{Размер в байтах} & \textit{Диапазон} \\
        \hline
        \lstinline|bool| & 1 & true/false \\
        \hline
        \lstinline|char| & 1 & \numrange{-128}{127} \\
        \hline
        \lstinline|unsigned char| & 1 & \numrange{0}{255} \\
        \hline
        \lstinline|short int| & 2 & \numrange{-32 768}{32 767} \\
        \hline
        \lstinline|unsigned short int| & 2 & \numrange{0}{65 535} \\
        \hline
        \lstinline|int| & 4 & \numrange{-2 147 483 648}{2 147 483 647} \\
        \hline
        \lstinline|unsigned int| & 4 & \numrange{0}{4 294 967 295} \\
        \hline
        \lstinline|long int| & 4 & \numrange{-2 147 483 648}{2 147 483 647} \\
        \hline
        \lstinline|unsigned long int| & 4 & \numrange{0}{4 294 967 295} \\
        \hline
        \lstinline|long long| & 8 & \num{-9 223 372 036 854 775 808} \ldots \newline \ldots \num{9 223 372 036 854 775 807} \\
        \hline
        \lstinline|unsigned long long| & 8 & \numrange{0}{18 446 744 073 709 551 615} \\
        \hline
        \lstinline|float| & 4 & $\pm 3.4\cdot 10^{\pm 38}$ (7 цифр) \\
        \hline
        \lstinline|double| & 8 & $\pm 1.7\cdot 10^{\pm 308}$ (15 цифр) \\
        \hline
    \end{tabular}
\end{table}



\subsubsection{Операции с переменными}
Стандартные арифметические операции над переменными:
\begin{enumerate}
    \item Операция присваивания (\lstinline{=}). Значение выражения присваивается переменной:
    \begin{lstlisting}
int a = 5;
int b = 42; // a = 5, b = 42
a = b; // a = 42, b = 42
    \end{lstlisting}
    \item Операция сложения (\lstinline{+}), вычитания (\lstinline{-}), умножения (\lstinline{*}), целочисленное деление (\lstinline{/}), взятие остатка по модулю (\lstinline{%}):
    \begin{lstlisting}
int a = 5;
int b = 3;

int l = 5 + 3 - 1; // l = 7
int c = a + b; // c = 8
int d = a - b; // d = 2
int k = a * b; // k = 15
int div = a / b; // div = 1
int mod = a % b; // mod = 2
    \end{lstlisting}

    Первоначально обрабатываются операции умножения, деления и взятия остатка по модулю. Если операций несколько, то выполняются слева направо. После этих операций по такому же принципу обрабатываются и операции сложения и вычитания.

    \item Операция \lstinline{()}. Выражения в скобках оцениваются в первую очередь. В случае вложенных скобок вычисляется значение во внутренних скобках. Если есть одинаковые скобки <<одного уровня>> --- они вычисляются слева направо.
\end{enumerate}

Существуют короткие формы записей арифметических операций:
\begin{lstlisting}
a += b --> a = a + b
a -= b --> a = a - b
a *= b --> a = a * b
a /= b --> a = a / b
a %= b --> a = a % b
\end{lstlisting}

В \texttt{С++} существуют операции инкремента (\lstinline{++}) и декремента (\lstinline{--}). Данные операции увеличивают или уменьшают значение переменной на 1. Различают префиксную форму (\lstinline{++a}\textbackslash \lstinline{--b}) и суффиксную (постфиксную) форму (\lstinline{a++}\textbackslash \lstinline{b--}). Различие между формами в том, что при использовании префиксной формы значение переменной изменяется сразу же, как программа выполняет данную операцию; в суффиксной --- программа сперва использует старое значение, а потом изменяет его.

Операции сравнения над переменными:
\begin{enumerate}
    \item Операция равенства (\lstinline{==}). Принимает значение \textbf{true}, если оба выражения равны, в противном случае --- \textbf{false}. \textit{Не следует путать с присваиванием значения переменной!!!}

    \begin{lstlisting}
bool result = 4 % 2 == 0; // true

int a = 5;
int b = 2;
bool newResult = a == b; // false
    \end{lstlisting}

    \item Операция неравенства (\lstinline{!=}). Принимает значение \textbf{true}, если оба выражения различны, в противном случае --- \textbf{false};
    \item Операции отношения (\lstinline{>}, \lstinline{<}, \lstinline{>=}, \lstinline{<=}). Возвращают значение типа \textit{bool} в зависимости от своей функции:

    \begin{lstlisting}
bool a = 5 > 3; // true
bool b = 4 < 2; // false
bool c = 6 <= 6; // true
bool d = 5 >= 2; // true
    \end{lstlisting}
\end{enumerate}

Логические операции:

\begin{enumerate}
    \item Логическое отрицание (\lstinline{!}) --- Инвертирует логическое значение;
    \item Логическое умножение, И (\lstinline{&&}). Если первое выражение равно \textit{false}, то тогда вычисление второго выражения не производится;
    \item Логическое сложение, ИЛИ (\lstinline{||}). Первое выражение вычисляется всегда, второе вычисляется в том случае, если первое выражение принимает значение \textit{fasle}.
\end{enumerate}

Прочие операции:
\begin{itemize}
    \item \lstinline{?:} --- тернарный оператор, подробнее в~\ref{subsubsec:ternaryOperator};
    \item \lstinline{[]} --- взятие элемента массива по индексу, подробнее в~\ref{subsec:Arrays}.
\end{itemize}

\subsubsection{Приоритеты и ассоциативность операций}

\begin{table}[ht]
    \centering
    \begin{tabular}{|>{\centering\arraybackslash}m{3cm}|>{\centering\arraybackslash}m{5cm}|>{\centering\arraybackslash}m{5cm}|}
        \hline
        \textit{Операции} & \textit{Ассоциативность} & \textit{Тип} \\
        \hline
        \lstinline!()! \lstinline![]! & слева направо & наивысший приоритет \\
        \hline
        \lstinline!++! \lstinline!--! & слева направо & унарные (постфиксные) \\
        \hline
        \lstinline!++! \lstinline!--! \lstinline|!| & справа налево & унарные (префиксные) \\
        \hline
        \lstinline!*! \lstinline!/! \lstinline!%! & слева направо & мультипликативные \\
        \hline
        \lstinline!+! \lstinline!-! & слева направо & аддитивные \\
        \hline
        \lstinline!<<! \lstinline!>>! & слева направо & передача\textbackslash извлечение из потока \\
        \hline
        \lstinline!<! \lstinline!<=! \lstinline!>! \lstinline!>=! & слева направо & отношения \\
        \hline
        \lstinline!==! \lstinline|!=| & слева направо & равенства \\
        \hline
        \lstinline!&&! & слева направо & логическое И \\
        \hline
        \lstinline!||! & слева направо & логическое ИЛИ \\
        \hline
        \lstinline!?:! & справа налево & условная \\
        \hline
        \lstinline!=! \lstinline!+=! \lstinline!-=! \lstinline!*=! \lstinline!/=! \lstinline!%=! & справа налево & присваивания \\
        \hline
        \lstinline!,! & слева направо & запятая \\
        \hline
    \end{tabular}
\end{table}

\subsection{Условный оператор \texttt{if}, тернарный оператор, оператор \texttt{switch}}
\subsubsection{Условный оператор \texttt{if}}
Общая структура условного оператора:
\begin{lstlisting}
if (expression_1)
    statement_1
[else
    statement_2]
\end{lstlisting}

\lstinline|expression| --- логическое выражение; \lstinline|statement\_X| --- выполняемая операция.

Начало условия задается ключевым словом \lstinline|if|, после которого в круглых скобках идет логическое выражение типа \lstinline|bool|. Наличие круглых скобок вне выражения обязательно, в противном случае компилятор выкинет ошибку. После выражения идут операции, которые следует выполнять в том случае, если выражение в круглых скобках истинно. Если операций несколько, то необходимо поставить фигурные скобки, в противном случае ими можно обойтись.

Пример условия:

\begin{lstlisting}
int a = 13;
if (a < 16)
    a = a + 2;

// a = 15

int count = 2;
if (count > 3)
{
    count++;
    age += 5;
}

// count = 2, age = 15
\end{lstlisting}

Если необходимо выполнить набор действий в том случае, когда выражение ложно, то необходимо поставить ключевое слово \lstinline|else| и перечислить набор операций (если операция одна, то фигурные скобки можно опустить, в противном случае они обязательны). Если условие не предполагает действий в том случае, если выражение в скобках ложно, то нет необходимости ставить \lstinline|else|.

Пример:
\begin{lstlisting}
int a = 5;
int b = 0;

if (a % 2 == 0)
    b++;
else
{
    a--;
    b--;
}

// b = -1
\end{lstlisting}

Допускается использования нескольких блоков \lstinline|if| \ldots \lstinline|else|:

\begin{lstlisting}
int age = 15;

if (age <= 3)
    ...
else if (age > 3 && age <= 7)
    ...
else if (age > 7 && age <= 18)
    ...
else
    ...
\end{lstlisting}

\subsubsection{Тернарный оператор}
\label{subsubsec:ternaryOperator}
Тернарный оператор возвращает одно из двух выражений в зависимости от результата логического выражения.

Структура тернарного оператора:

\begin{lstlisting}
expression ? statement_1 : statement_2;
\end{lstlisting}

Оператор сначала вычисляет выражение \lstinline{expression}.
Если это выражение принимает значение \lstinline|true|,
то результатом выполнения всего оператора будет \lstinline|statement_1|,
а в противном случае~---~\lstinline|statement_2|.

Пример:
\begin{lstlisting}
int a = 3, b = 5;
int min = a < b ? a : b;
// a = 3
\end{lstlisting}

\subsubsection{Оператор \texttt{switch}}
Структура оператора \lstinline|switch|:
\begin{lstlisting}
switch ( expression )
{
    case constant-expression : statement
    [default   : statement]
}
\end{lstlisting}

\lstinline|expression| --- выражение, которое можно привести к целочисленному типу; \lstinline|constant-expression| --- константное значение, если выражение будет иметь данное значение, то произойдет обработка блока \lstinline|statement|. Если ни один из \lstinline|case|'ов не сработал, то вычисляется \lstinline|statement| в блоке \lstinline|default|, однако данный блок не обязателен к указанию и может быть опущен --- в таком случаю программа перейдет к выполнению следующих инструкций после оператора \lstinline|switch|.

Пример использования:
\begin{lstlisting}
int month = 4;
switch (month)
{
    case 1:
        ...
        break;
    case 2:
        ...
        break;
    ...
    case 4:
        ...
        break;
    ...
    case 12:
        ...
        break;
    default:
        ...
}
\end{lstlisting}

В данном примере будут выполняться инструкции, которые идут в блоке \lstinline|case 4:|. Обратите внимание, что после окончания инструкций в блоке стоит оператор \lstinline|break|~---~он необходим для прекращения работы оператора \lstinline|switch| после выполнения команд в блоке, в противном случае оператор \lstinline|switch| будет продолжать свою работу!

Существует возможность указания нескольких \lstinline|case|'ов для одного и того же блока команд:

\begin{lstlisting}
char answer = 'y';
switch (answer)
{
    case 'y':
    case 'Y':
        ...
        break;
    case 'n':
    case 'N':
        ...
        break;
}
\end{lstlisting}

\subsection{Циклы \texttt{while}, \texttt{do-while}, \texttt{for}}
\subsubsection{Цикл с предусловием (\texttt{while})}
Структура цикла \lstinline|while|:
\begin{lstlisting}
while (expression)
    statement
\end{lstlisting}

\lstinline|expression| --- выражение типа \lstinline|bool|, задает условие выхода из цикла; \lstinline|statement|~---~набор операций, которые будут выполняться в цикле (если операций больше одной, то нужно заключить их в фигурные скобки, в противном случае делать данное действие необязательно). Цикл \lstinline|while| будет выполняться до тех пор, пока \lstinline|expression| будет равно \lstinline|true|. Программа может не войти в тело цикла (то есть выполнить операции в нем), если изначально \lstinline|expression| будет равен \lstinline|false|.

Пример:

\begin{lstlisting}
int a = 0;
int b = 0;
while (a < 10)
{
    b *= a*a;
    a++;
}
\end{lstlisting}

Цикл \lstinline|while| можно досрочно завершить при помощи оператора \lstinline|break| (управление программы перейдет к следующей инструкции):

\begin{lstlisting}

char answer = 'v';

while (a != 'y')
{
    ...

    if (answer == 'q')
        break;

    ...
}

\end{lstlisting}

В данном примере выход из цикла произойдет в двух случаях: пока переменная \textit{answer} не примет значение \textit{y}, либо значение \textit{q}. Во втором случае все операции после данного условия не будут выполнены и произойдет выход из цикла.

Если использовать оператор \lstinline|continue| в цикле, то все операции выполняться также не будут, однако цикл не будет завершен и его выполнение будет продолжено с самой первой операции в цикле.

\subsubsection{Цикл с постусловием (\texttt{do-while})}
Структура цикла \lstinline|do-while|:
\begin{lstlisting}
do
    statement
while (expression);
\end{lstlisting}

\lstinline|expression| --- выражение типа \lstinline|bool|, задает условие выхода из цикла; \lstinline|statement|~---~набор операций, которые будут выполняться в цикле (если операций больше одной, то нужно заключить их в фигурные скобки, в противном случае делать данное действие необязательно). Цикл \lstinline|do-while| будет выполняться до тех пор, пока выражение \lstinline|expression| будет равно \lstinline|true|. В данном цикле блок \lstinline|statement| будет выполняться в первую очередь, условие выхода из цикла будет проверяться после выполнения операций. Операторы \lstinline|break| и \lstinline|continue| имеют такой же эффект, что и в цикле \lstinline|while|.

Пример:
\begin{lstlisting}
int a = 0;
do
{
    a++;
}
while (a < 10);
\end{lstlisting}

\subsubsection{Цикл со счетчиком (\texttt{for})}
Конструкция цикла \lstinline|for|:
\begin{lstlisting}
for (init-expression ; cond-expression ; loop-expression)
    statement
\end{lstlisting}

Обозначения:
\begin{itemize}
    \item \lstinline|init-expression| --- набор операций, которые выполняются один раз \textbf{до} начала цикла, чаще всего используется для инициализации переменных, которые будут использоваться в цикле для подсчета итераций;
    \item \lstinline|cond-expression| --- выражение типа \lstinline|bool|, которое задает условие выхода из цикла --- проверяется перед каждой итерацией цикла (в том числе и перед самой первой), если \textit{cond-expression} равное \lstinline|false|, то цикл завершается;
    \item \lstinline|loop-expression| --- набор операций, который выполняется после итерации цикла, обычно используется для изменения переменных, который подсчитывают количество итераций;
    \item \lstinline|statement| --- набор операций, которые будут производится в каждой итерации цикла (если операций больше одной, то нужно заключить их в фигурные скобки, в противном случае делать данное действие необязательно).
\end{itemize}

Пример:

\begin{lstlisting}
int sum = 0;
const int lim = 10;

for (int i = 0; i < lim; ++i)
    sum += i;
\end{lstlisting}

В блоках \lstinline|init-expression| и \lstinline|loop-expression| может быть несколько выражений, разделенных запятой:
\begin{lstlisting}
int sum = 0;

for (int i = 0, j = 0; i + j < lim; ++i, ++j)
    sum += i + j;
\end{lstlisting}

В блоке \lstinline|loop-expression| переменные, ответственные за количество
итераций, могут изменяться по-разному:
\begin{lstlisting}
for (int i = 10; i > 0; --i)
{
    ...
}

for (int k = 0; k > 200; k += 2)
{
    ...
}
\end{lstlisting}

Операторы \lstinline|break| и \lstinline|continue| имеют такой же эффект, что и в циклах \lstinline|while| и \lstinline|do-while|.

\subsection{Массивы}
\label{subsec:Arrays}
\subsubsection{Одномерные массивы}
Общая структура:

\lstinline|Type arrayName[arraySize];|

\lstinline|Type| --- тип элементов массива
(массив может состоять из целых чисел, дробных, символов, \ldots);
\lstinline|arrayName|~---~название массива; \lstinline|arraySize|~---~размер
массива. Индексы, по которым можно обращаться к элементам массива,
лежат в диапазоне~$\left[ 0 \ldots \text{arraySize} - 1 \right]$.

Обращение к элементу одномерного массива по индексу --- \texttt{arrayName[index]}.

Инициализация одномерного массива:
\begin{itemize}
    \item Использовать инициализатор \{\} при объявлении массива:

    \lstinline|int costs[4] = {100, 200, 300, 400};|

    В таком случае можно не указывать размер массива:

    \lstinline|int costs[] = {2, 4, 6};|

    \item Инициализировать массив при помощи циклов:
    \begin{lstlisting}
const int arraySize = 5;
int myArray[arraySize];

for (int i = 0; i < arraySize; ++i)
    myArray[i] = ...;
    \end{lstlisting}
\end{itemize}

Пример инициализированного массива:

\begin{table}[h]
    \begin{tabular}{|c|c|c|c|c|}
      \hline
      Индекс массива & 0 & 1 & 2 & 3 \\
      \hline
      Значение массива & 34 & -4 & 11 & 64 \\
      \hline
    \end{tabular}
\end{table}

Рекомендуется при работе с одномерными массивами инициализировать их, т.к. в противном случае в элементах массивов будет записан <<мусор>>!

\subsubsection{Двумерные массивы}
Общая структура:

\lstinline|Type arrayName[arraySizeRow][arraySizeColumn];|

\lstinline|Type| --- тип массива (массив может состоять из целых чисел, дробных, символов, \ldots); \lstinline|arrayName|~---~название массива; \lstinline|arraySizeRow|~---~количество одномерных массивов, \lstinline|arraySizeColumn|~---~размер каждого одномерного массива. Количество одномерных массивов лежит в диапазоне $\left[ 0 \ldots \text{arraySizeRow} - 1 \right]$, размерность одномерных массивов~---~$\left[ 0 \ldots \text{arraySizeColumn} - 1 \right]$. Обращение к элементу двумерного массива по индексам --- \texttt{arrayName[\textit{indexRow}][\textit{indexColumn}]}.

Инициализация двумерного массива:
\begin{itemize}
    \item Использовать инициализатор \{\} при объявлении массива:
    \begin{lstlisting}
int example[][5] =
{
    {100, 200, 300, 400},
    {200, 600, 1000, 1400},
    {1000, 800, 600, 400}
}
    \end{lstlisting}

    Обратите внимание, что указывается размерность одномерных массивов, но не количество этих массивов!

    \item Инициализировать массив при помощи циклов:
    \begin{lstlisting}
const int rowSize = 3;
const int columnSize = 5;
int myArray[rowSize][columnSize];

for (int i = 0; i < rowSize; ++i)
    for (int j = 0; j < columnSize; ++j)
        myArray[i][j] = ...;
    \end{lstlisting}
\end{itemize}

Пример инициализированного массива:
\begin{table}[h]
    \begin{tabular}{|c|c|c|c|c|}
        \hline
          & 0 & 1 & 2 & 3 \\
        \hline
        0 & 24 & 25 & 23 & 22 \\
        \hline
        1 & 1 & 2 & 3 & 4 \\
        \hline
        2 & 3 & 5 & 7 & 11 \\
        \hline
        3 & 10 & 15 & 20 & 25 \\
        \hline
        4 & 11 & 22 & 33 & 44 \\
        \hline
        5 & 0 & 2 & 4 & 6 \\
        \hline
    \end{tabular}
\end{table}

\subsection{Строки}
\subsubsection{Инициализация строк}
Для включения строк \texttt{C++} необходимо подключить заголовочный файл \lstinline|<string>|. Сам класс находится в пространстве имен \textbf{std}. Объект \lstinline|string| можно инициализировать:

\begin{enumerate}
    \item С помощью конструктора:
    \begin{lstlisting}
std::string text("Hello");
std::string anotherText(3, 'p'); // "ppp"
std::string emptyString;
    \end{lstlisting}

    В последнем случае \lstinline|std::string emptyString| вызывается конструктор по умолчанию, который создает пустую строку
    \item При помощи оператора \lstinline{=}, который вызывает конструктор:
    \begin{lstlisting}
std::string month = "August";
std::string emptyString = "";
    \end{lstlisting}
\end{enumerate}

Числа нельзя присваивать напрямую строке:
\begin{lstlisting}
std::string error = 'c'; // ERROR
std::string anotherError(8); // ERROR
\end{lstlisting}

Оператор \lstinline|>>| потока \lstinline|std::cin| может также работать со строками:
\begin{lstlisting}
std::string userAnswer;
std::cin >> userAnswer;
\end{lstlisting}

\subsubsection{Взятие символа строки (\texttt{[], at()}), размер строки (\texttt{length(), empty()})}
После инициализации строки можно обращаться к отдельным символам строки:
\begin{enumerate}
    \item Через оператор \lstinline|[]|:
    \begin{lstlisting}
std::string student = "Vasya";
std::cout << student[0]; // 'V'
std::cout << student[4]; // 'a'

student[2] = 'a';
std::cout << student; // "Vaaya"
    \end{lstlisting}
    \item Через метод \lstinline|at(int index)| класса \lstinline|string|:
    \begin{lstlisting}
std::string student = "Vasya";
std::cout << student.at(0); // 'V'
std::cout << student.at(3); // 'y'

student.at(0) = 'M';
std::cout << student; // "Masya"
    \end{lstlisting}
\end{enumerate}

Индексы, по которым можно обращаться к элементам строки,
лежат в диапазоне~$\left[ 0 \ldots \text{stringLength} - 1 \right]$.

Размер строки можно узнать с помощью метода \lstinline|length()|:
\begin{lstlisting}
std::string myString = "abcd";
std::cout << myString.length(); // 4
\end{lstlisting}

Для того, чтобы узнать, является ли строка пустой, можно вызывать метод \lstinline|empty()|:
\begin{lstlisting}
std::string nonEmptyString("qew12314");
std::string emptyString;

std::cout << nonEmptyString.empty() << std::endl; // 0
std::cout << emptyString.empty() << std::endl; // 1
\end{lstlisting}

\subsubsection{Присваивание (\texttt{assign()}) и конкатенация строк(\texttt{append()})}

Возможные варианты присваивания:
\begin{lstlisting}
std::string str1("test");
std::string str2,str3,str4;

str2 = str1; // str2 = "test";
str3.assign(str1); // str3 = "test";
\end{lstlisting}

Метод \lstinline|assign| можно использовать, когда необходимо присвоить \texttt{подстроку}. В таком случае метод принимает дополнительные параметры:

\lstinline|assign(source, startIndex, length)|

\textit{source}~---~строка типа \lstinline|string|, из которой будет браться подстрока, \lstinline|startIndex|~---~начальный индекс, с которого будет начинаться копирование, \lstinline|length|~---~количество символов, которые будут скопированы.
Необходимо, чтобы выполнялось условие~$length \leqslant source.length() - startIndex$.

Пример:
\begin{lstlisting}
std::string str1 = "abcd";
std::string str2;
str2.assign(str1, 2, 2);
std::cout << str2; // str2 = "cd";
\end{lstlisting}

Возможные варианты конкатенации:
\begin{lstlisting}
std::string str1("abcd");
std::string str2;

str2 += str1; // str2 = "abcd";
std::string str3(str1 + str2); // str5 = "abcdabcd"

std::string str4 = "zyx";
str4.append(str1); // str4 = "zyxabcd";
\end{lstlisting}

Метод \lstinline|append| можно использовать, когда необходима конкатенация \texttt{подстроки}. В таком случае метод принимает дополнительные параметры:

\lstinline|append(source, startIndex, length)|


\lstinline|source|~---~строка типа \lstinline|string|, из которой будет браться подстрока, \lstinline|startIndex|~---~начальный индекс, с которого будет начинаться копирование, \lstinline|length|~---~количество символов, которые будут скопированы.
Необходимо, чтобы выполнялось условие~$length \leqslant source.length() - startIndex$.

Пример:
\begin{lstlisting}
std::string str1 = "abcdfg";
std::string str2 = "111";
str2.append(str1, 1, 2);

std::cout << str2; // str2 = "111bc"
\end{lstlisting}

\subsubsection{Сравнение строк (\texttt{compare()})}
Строки сравниваются лексикографически. Сравнение происходит по первому несовпадающему символу, в противном случае сравниваются длины строк.
Строки можно сравнивать с помощью операторов \lstinline|==|, \lstinline|>|, \lstinline|<|:
\begin{lstlisting}
std::string str1("abcd");
std::string str2("Hello");
std::string str3(str2);
std::string str4("zxc");
std::string str5("zxcd");

std::cout << (str1 == str2) << endl; // false (0)
std::cout << (str2 == str3) << endl; // true (1)
std::cout << (str1 < str4) << endl; // true (1)
std::cout << (str1 > str2) << endl; // true (1)
std::cout << (str4 > str5) << endl; // false (0)
\end{lstlisting}

Сравнивать строки между собой можно с помощью метода \lstinline|compare()|. Функция возвращает 0, если строки одинаковы; значение меньше 0, если строка лексикографически меньше или все сопоставимые символы одинаковы, но строка меньше; значение больше 0, если строка лексикографически больше или все сопоставимые символы одинаковы, но строка больше.

Пример:

\begin{lstlisting}
std::string str1("abcd");
std::string str2("Hello");
std::string str3("zxc");
std::string str4("abcdfff");

std::cout << str1.compare(str1) << endl; // 0
std::cout << str1.compare(str2) << endl; // 1
std::cout << str1.compare(str3) << endl; // -1
std::cout << str1.compare(str4) << endl; // -1
\end{lstlisting}

Метод \lstinline|compare()| может сравнивать подстроку вызываемой строки с другой строкой:
\begin{lstlisting}
compare(startIndex, length, anotherString)
\end{lstlisting}

\lstinline|startIndex|~---~начальный индекс, с которого будет начинаться копирование, \lstinline|length|~---~количество символов, которые будут скопированы, \lstinline|anotherString|~---~строка, с которой будет сравниваться подстрока. Необходимо подбирать \lstinline|startIndex| и \lstinline|length| так, чтобы подстрока была корректной.

Пример:

\begin{lstlisting}
std::string str1("abcdefgh");
std::string str2("abcd");

std::cout << str1.compare(0, 4, str2) << endl; // 0
std::cout << str2.compare(3, 1, str1) << endl; // 1
\end{lstlisting}

Метод \lstinline|compare| может сравнивать подстроку вызываемой строки с подстрокой другой строки
\begin{lstlisting}
compare(startIndex, length, anotherString, anotherStartIndex, anotherLength)
\end{lstlisting}

\lstinline|startIndex|~---~начальный индекс, с которого будет начинаться копирование, \lstinline|length|~---~количество символов, которые будут скопированы, \lstinline|anotherString|~---~строка, с которой будет сравниваться подстрока, \lstinline|anotherStartIndex| и \lstinline|anotherLength| аналогичны по своим функциям с \lstinline|startIndex| и \lstinline|length|.

\subsubsection{Подстрока (\texttt{substr})}
С помощью метода \lstinline{substr} можно брать подстроки оригинальной строки:

\lstinline|substr(startIndex, length)|

Пример использования:
\begin{lstlisting}
std::string first("abcdefg");
std::string subFirst = first.substr(3, 4); // "defg"
\end{lstlisting}

\subsubsection{Поиск подстрок и символов (\texttt{find, rfind, find\_first\_of, \\find\_last\_of, find\_first\_not\_of, find\_last\_not\_of})}

Для поиска подстрок необходимо использовать методы \lstinline|find()|, который ведет поиск слева направо, и \lstinline|rfind()|, который ищет справа налево. На вход методы принимают подстроку типа \lstinline|std::string| и возвращают индекс начальной позиции этой подстроки. Если данной подстроки нет, то возвращается \lstinline|std::string::npos|.

Примеры:
\begin{lstlisting}
std::string str("abcdefghiabc");

std::cout << str.find("abc") << std::endl; // 0
std::cout << str.find("fg") << std::endl; // 5
std::cout << str.rfind("abc") << std::endl; // 9
std::cout << str.rfind("zzz") << std::endl; // string::npos
\end{lstlisting}

Методы \lstinline|find_first_of(), find_last_of(), find_first_not_of(), find_last_not_of()| принимает на вход набор символов в виде \lstinline|std::string| и возвращают индекс первого\textbackslash последнего вхождения\textbackslash отсутствия в зависимости от своей функции, в противном случае методы возвращают \lstinline|std::string::npos|.

Назначения методов:
\begin{itemize}
    \item \lstinline|find_first_of|~---~возвращает индекс первого вхождения любого из символов параметра в строке;
    \item \lstinline|find_last_of|~---~возвращает индекс последнего вхождения любого из символов параметра в строке;
    \item \lstinline|find_first_not_of|~---~возвращает индекс первого вхождения любого символа \textbf{не} из символов параметра в строке;
    \item \lstinline|find_last_not_of|~---~возвращает индекс последнего вхождения любого символа \textbf{не} из символов параметра в строке.
\end{itemize}

Примеры:
\begin{lstlisting}
std::string str("abcdefghiabc");

std::cout << str.find_first_of("tree") << std::endl; // 4
std::cout << str.find_last_of("cat") << std::endl; // 11
std::cout << str.find_first_not_of("father") << std::endl; // 1
std::cout << str.find_last_not_of("car") << std::endl; // 10
\end{lstlisting}

\subsubsection{Удаление символов из строки (\texttt{erase(), clear()})}

При помощи метода \lstinline|erase()| можно удалить $n$--ое количество символов:

\lstinline|erase(index, length)|

\lstinline|index| указывает индекс с которого необходимо начать удаление, \lstinline|length| указывает сколько символов необходимо удалить. Если не указывать параметр \lstinline|length|, то будут удаление будет происходить до конца строки.

Примеры:
\begin{lstlisting}
std::string test("abcdefgh");
test.erase(3);
std::cout << test << std::endl; // test = "abc"

test = "12345678";
test.erase(1, 6);
std::cout << test << std::endl; // test = "18";
\end{lstlisting}

Если необходимо полностью очистить строку, то необходимо использовать метод \lstinline|clear()|:
\begin{lstlisting}
std::string test("1234abcd");
std::cout << test.empty() << std::endl; // 0

test.clear();
std::cout << test.empty() << std::endl; // 1
\end{lstlisting}

\subsubsection{Замена символов в строке (\texttt{replace()})}

Для замены символов в строке необходимо использовать метод \lstinline|replace()|:

\lstinline|replace(index, length, replaceString)|

\lstinline|index| указывает с какого индекса в строке необходимо произвести замену, \lstinline|length| указывает сколько символов необходимо заменить, \lstinline|replaceString| является строкой, на которую происходит замена.

Пример:
\begin{lstlisting}
std::string test("1234abcd");
test.replace(4, 4, "5678");
std::cout << test << std::endl;
// test = "12345678"
\end{lstlisting}

\subsubsection{Вставка символов в строку(\texttt{insert()})}
Для вставки символов в строку необходимо использовать метод \lstinline|insert()|:

\lstinline|insert(index, insertString)|

\lstinline|index| указывает индекс, с которого необходимо сделать вставку, \lstinline|insertString| является строкой, которую необходимо вставить.

Пример:
\begin{lstlisting}
std::string test("14");
test.insert(1, "23");
std::cout << test << std::endl; // "1234"
\end{lstlisting}

\subsection{Функции}
\subsubsection{Синтаксис}

В языке \texttt{C++} можно создавать пользовательские функции, которые можно использовать в своих программах. Функции помогают структурировать код, делать его более понятным. Если в коде присутствуют одинаковые куски кода, то их имеет смысл вынести в функции.

Синтаксис пользовательских функций:
\begin{lstlisting}
    FunType functionName(ParamType1 paramName1, ..., ParamTypeN paramNameN)
    {
        ...
    }
\end{lstlisting}

\lstinline|FunType|~---~тип возвращаемого значения функции, \lstinline|functionName|~---~наименование функции, \lstinline|ParamType|~---~тип параметра, \lstinline|paramName|~---~наименование параметра.

Пример функции, которая принимает в качестве параметра целочисленное число и возвращает его квадрат:

\begin{lstlisting}
    long long sqr(const int value)
    {
        return value * value;
    }
\end{lstlisting}

Функция может возвращать любой базовый тип в С++, любое пользовательское значение при помощи ключевого слова \lstinline|return|. В таком случае произойдет выход из функции и передача значения. Также функция может ничего не возвращать, в таком случае необходимо в качестве типа использовать ключевое слово \lstinline|void|. В случае, если использовать \lstinline|return| в функции, которая возвращает значение \lstinline|void|, то функция просто завершит свою работу и передаст управление программе дальше.

Пример функции, которая выводит строку на экран в том случае, если она не пустая:

\begin{lstlisting}
    void printString(const std::string text)
    {
        if (!text.empty())
            std::cout << text << std::endl;
    }
\end{lstlisting}

\subsubsection{Прототипы функций}
Функция вместе с реализацией должна находиться раньше, чем происходит её вызов, в противном случае произойдет ошибка компиляции, так как компилятор не будет знать о существовании данной функции.

\begin{lstlisting}
    long long sqr(const int value)
    {
        ...
    }
    
    int main()
    {
        ...
        long long a = sqr(5);
        ...
        return 0;
    }
\end{lstlisting}

Однако существует способ разделить объявление функции от её реализации. Для этого необходимо использовать \texttt{прототипы функций}. Для этого необходимо привести до использования функции её \texttt{сигнатуру}~---~тип возвращаемого значения функции, имя функции и её параметры. Потом необходимо привести \texttt{реализацию} функции:

\begin{lstlisting}
    long long sqr(const int value); // signature
    
    int main()
    {
        ...
        long long a = sqr(5); // function call
        ...
        return 0;
    }
    
    long long sqr(const int value) // realization
    {
        return value * value;
    }
\end{lstlisting}

\subsubsection{Передача аргументов в функции по значению и по ссылке}
Аргументы в функцию можно передавать по значению (в таком случае на стороне функции произойдет полное копирование аргумента, что при больших типах данных может замедлить работу, но зато изменение аргумента в функции не повлияет на переменную, которая была передана извне функции), либо по ссылке (изменение переменной будет произведено как и в самой функции, так и извне неё). Способ передачи аргумента задается в \texttt{сигнатуре} функции.

Аргумент по значению:
\begin{lstlisting}
    void incCopy(int value)
    {
        ++value;
    }
    
    int main()
    {
        int a = 1;
        inc(a);
        std::cout << a << std::endl; // a = 1; 
    }
\end{lstlisting}

Аргумент по ссылке:
\begin{lstlisting}
    void incRef(int &value)
    {
        ++value;
    }

    int main()
    {
        int a = 1;
        inc(a);
        std::cout << a << std::endl; // a = 2;
    }
\end{lstlisting}

\subsubsection{Аргументы по умолчанию}
У параметров пользовательских функций можно выставлять аргументы по умолчанию, которые будут применяться, если в аргументы функции ничего не будет передаваться при вызове. Данные аргументы должны быть самыми правыми аргументами в списке параметров функции. Если вызывается функция с двумя или более аргументами по умолчанию и если пропущенный аргумент не является самым правым в списке аргументов, то все аргументы справа от пропущенного тоже пропускаются. 

Аргументы по умолчанию обычно указываются в прототипе функции при помощи оператора \lstinline|=|, в реализации функции указывать значения по умолчанию не нужно.

Пример функции, которая подсчитывает площадь прямоугольника, в том случае, если указана ширина, в противном случае подсчитывается площадь квадрата по переданной длине:

\begin{lstlisting}
    double getSquare(double height, double width = -1.0);
    
    int main()
    {
        std::cout << getSquare(5.0, 6.0) << std::endl; // 30.0
        std::cout << getSquare(10.0) << std::endl // 100.0
    }
    
    double getSquare(double height, double width)
    {
        return weight == -1.0 ? height * height : height * width;
    }
\end{lstlisting}

\subsection{Структуры}
\subsubsection{Синтаксис объявления}
Структуры~---~пользовательский тип данных, который позволяет группировать переменные с разными типами под одним именем. Объявлять структуры необходимо в самом начале файла после объявления \lstinline|#include| и использования пространств имен (\lstinline|using namespace ...|), но до функций. Делается это для того, чтобы каждая функция могла <<знать>> о существовании данной структуры.

Синтаксис структуры:
\begin{lstlisting}
    struct structName
    {
        Type variableName1;
        Type variableName2;
        ...
        Type variableNameN;
    };
\end{lstlisting}

\lstinline|structName|~---~пользовательское имя структуры, \lstinline|Type|~---~тип переменной в структуре, \lstinline|variableName|~---~имя переменной (оно же поле структуры).

Пример структуры:
\begin{lstlisting}
    struct Person
    {
        std::string name;
        std::string sirname;
        unsigned int age;
    };
\end{lstlisting}

Создание экземпляра структуры:
\lstinline{structName ourStruct;}

\subsubsection{Конструкторы структур}
Для структуры можно указывать специальный метод~---~\texttt{конструктор}, с помощью которого при инициализации экземпляра структуры можно определять поля структуры:
\begin{lstlisting}
    struct Person
    {
        std::string name;
        std::string surname;
        unsigned int age;
        // constructor
        Person(std::string personName, std::string personSurname, unsigned int personAge)
        {
            name = personName;
            sirname = personSurname;
            age = personAge;
        }
    };
\end{lstlisting}

При объявлении конструктора нужно следовать следующим правилам:
\begin{itemize}
    \item Конструктор должен иметь такое же имя, что и название структуры;
    \item Перед конструктором не должно быть типа (даже \texttt{void'}а).
\end{itemize}

Так как конструктор является функцией, то оно может принимать на вход \textit{n}-ое количество параметров, с помощью которых можно инициализировать поля структуры. Также структура может иметь несколько конструкторов с разными параметрами.

Так же структура может иметь конструктор по умолчанию, который вызывается, если не был вызван ни один другой конструктор. Конструкторы по умолчанию используются для того, чтобы поля не были инициализированы <<мусором>>.

Пример конструктора по умолчанию:
\begin{lstlisting}
    struct Person
    {
        std::string name;
        std::string surname;
        unsigned int age;

        // Default constructor
        Person()
        {
            name = "";
            surname = "";
            age = 0;
        }
    };
\end{lstlisting}


При инициализации экземпляра структуры необходимо передать параметры в конструктор:
\begin{lstlisting}
    int main()
    {
        ...
        Person man("Andrey", "Popov", 16);
        ...
    }
\end{lstlisting}

Если передавать на данный момент времени нечего, а необходимы корректно определенные поля, то необходимо использовать конструктор по умолчанию:
\begin{lstlisting}
    int main()
    {
        ...
        Person man();
        ...
    }
\end{lstlisting}

Также имеется возможность просто объявить экземпляр структуры, но в таком случае поля экземпляра структуры будут неинициализированными!

\begin{lstlisting}
    int main()
    {
        ...
        Person man;
        ...
    }
\end{lstlisting}

\subsubsection{Обращение к полям структуры}

После того, как экземпляр структуры создан, можно обращаться к его полям и изменять их. Чтобы получить доступ к полю, необходимо использовать оператор~\lstinline|.| к экземпляру структуры:

\begin{lstlisting}
    int main()
    {
        Person man();

        std::string name = "";
        std::string surname = "";
        unsigned int age = 0;

        std::cout << "Please, enter your name:" << std::endl;
        std::cin >> name;

        man.name = name;

        std::cout << "Please, enter your surname:" << std::endl;
        std::cin >> surname;

        man.surname = surname;

        std::cout << "Please, enter your age: " << std::endl;
        std::cin >> age;

        man.age = age;

        std::cout << "Hello, dear " << man.name << "! Your surname is " << man.surname << " and your age is " << man.age << std::endl;
    }
\end{lstlisting} 