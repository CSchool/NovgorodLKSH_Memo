\section{ООП (Основы)}
\subsection{Классы}
\subsubsection{Определение, объекты}

Объектно-ориентированное программирование (ООП)~---~парадигма программирования, в которой основными концепциями являются понятия объектов и классов. Если проводить аналогии с объектами из реальной жизни, то машина~---~это \emph{объект}, который построен по неким чертежам или схемам (\emph{класс}), которые описывают данный объект~---~его характеристики и возможности. У машины может быть свой вес, цвет, вместимость~---~всё это является \emph{свойствами} объектов. Значения этих свойств у различных машин (объектов) может различаться. Возможность ездить, подавать сигнал~---~это \emph{методы} объекта. Имея один чертеж (класс), можно создавать огромное количество машин (объектов), у которых будут в наличии все эти методы, причем не нужно будет задумываться о реализации этих методов для каждого из объектов!

Основные принципы ООП:
\begin{itemize}
    \item Инкапсуляция~---~данное свойство позволяет объединить данные и методы, работающие с ними, в классе, и скрыть детали реализации от пользователя;
    \item Наследование~---~свойство, позволяющее создать новый класс-потомок на основе уже существующего, при этом все характеристики класса родителя присваиваются классу-потомку;
    \item Полиморфизм~---~свойство классов, позволяющее использовать объекты классов с одинаковым интерфейсом без информации о типе и внутренней структуре объекта;
\end{itemize}

В \texttt{C++} класс~---~ это пользовательский тип данных, который может группировать в себе как переменные различных типов~---~\emph{свойства}, так и различные функции~---~\emph{методы}.

Класс можно объявить через ключевое слово \lstinline|class| и в фигурных скобках перечислить его свойства и методы:

\begin{lstlisting}
class Message
{
public:
    void showMessage(const std::string& text)
    {
        std::cout << "Hello user! " << text << std::endl;
    }
};

int main()
{
    Message worldMessage;
    worldMessage.showMessage("Welcome to our world!"); // "Hello user! Welcome to our world!"
    return 0;
}
\end{lstlisting}

В данном примере рассматривается класс \lstinline|Message|, который имеет метод \lstinline|showMessage|, который выводит на консоль текст, состоящий из строки \lstinline|"Hello user!"| и параметра \lstinline|text|, которые соединены между собой. В функции \lstinline|main| объявляется \emph{объект} \lstinline|worldMessage| класса \lstinline|Message|, у которого вызывается метод \lstinline|showMessage|.

Ещё один пример класса:
\begin{lstlisting}
class Course
{
private:
    std::string courseName;

public:
    Course(const std::string& course)
    {
        courseName = course;
    }

    void printCourseName()
    {
        std::cout << "Your course is " << courseName;
    }

    std::string& getCourseName()
    {
        return courseName;
    }
};

int main()
{
    Course firstCourse("C++");
    Course secondCourse("Geometry");

    firstCourse.printCourseName(); // "Your course is C++"
    secondCourse.printCourseName(); // "Your course is Geometry"

    std::cout << "My favorite course is " << firstCourse.getCourseName() << " and not so favorite course is " << secondCourse.getCourseName() << std::endl;

    // My favorite course is C++ and not so favorite course is Geometry

    return 0;
}
\end{lstlisting}

Данный класс \lstinline|Course| представляет собой предмет в школе. У него есть свойство \lstinline|courseName|, которое хранит в себе название курса; метод \lstinline|getCourseName|, который возвращает наименование курса; \lstinline|printCourseName| печатает название курса; \lstinline|Course| является \emph{конструктором} класса (подробнее см.~\ref{subsubsec:ContstructorDestructor}) и инициализирует \lstinline|courseName| параметром. В функции \lstinline|main| создается два курса \lstinline|firstCourse|, который инициализируется строкой \emph{C++} и \lstinline|secondCourse|, который инициализируется строкой \emph{Geometry}. Дальше выводятся названия этих функций при помощи метода \lstinline|printCourseName| и берутся напрямую названия курсов через \lstinline|getCourseName| для создания строки.

\subsubsection{Модификаторы доступа к членам класса}
В примере с классом \lstinline{Message} можно увидеть ключевое слово \lstinline|public|, которое указывает на модификатор доступа к этому методу. Всего существует 3 модификатора доступа в \texttt{C++}:

\begin{itemize}
    \item \lstinline|public|~---~все свойства и методы открыты для всех, т.е. через объект можно будет обращаться к методам, или изменять свойства (изменения могут происходить как внутри класса, так и снаружи);
    \item \lstinline|private|~---~свойства и методы в данной секции можно использовать только внутри класса, т.е. свойства изменяются только внутри \emph{этого} класса и методы можно вызывать только внутри \emph{этого} класса;
    \item \lstinline|protected|~---~доступ к свойствам и методам открыт как и внутри класса, так и внутри производных классов от изначального, снаружи свойства не изменяются, методы не вызываются;
\end{itemize}

Правила объявления секций доступа:
\begin{lstlisting}
class Test
{
    public:
        // public properties and methods
        int publicInt;
        void incPublicInt()
        {
            ++publicInt;
        }

    private:
        // private properties and methods
        int privateInt;
        void showMessage()
        {
            std::cout << "you can't invoke me from object!";
        }

    protected:
        // protected properties and methods
        // ...
};
\end{lstlisting}

Свойство \lstinline|publicInt| можно изменять как и внутри класса, так и снаружи через объект данного класса. Метод \lstinline|incPublicInt| также можно использовать как внутри класса, так и снаружи через объект. Доступ к свойству \lstinline|privateInt| и методу \lstinline|showMessage| через объект данного класса запрещен. Модификатор \lstinline|protected| необходим для ситуаций, когда происходит наследование классов.

\subsubsection{Get- и Set- функции}
Как правило, свойства в классах должны находится в \lstinline|private| секциях доступа. Однако существует возможность создать функции, которые позволяют получать значение (\lstinline|get|) и изменять эти значения (\lstinline|set|) через объекты классов.

Метод \lstinline|get| должна возвращать свойство класса:

\begin{lstlisting}
class BankAccount
{
private:
    double cash;
public:
    double getCash()
    {
        return cash;
    }
};
\end{lstlisting}

Метод \lstinline|set| присваивает определённому свойству значение параметра, переданного в метод. При этом можно осуществлять проверку значения параметра, чтобы не произошло некорректной работы в дальнейшем:

\begin{lstlisting}
class BankAccount
{
private:
    double cash;
public:
    void setCash(double newCash)
    {
        if (newCash >= 0)
            cash = newCash;
    }
};
\end{lstlisting}

\subsubsection{Конструктор и деструктор}
\label{subsubsec:ContstructorDestructor}
\emph{Конструктор}~---~функция, которая вызывается в самом начале <<жизни>> объекта класса (сразу же после инициализации объекта в памяти) и необходима для подготовки объекта к работе: инициализация свойств и вызовов методов. \emph{Деструктор}~---~функция, которая вызывается перед тем, как будет удален объект из памяти и предназначен прежде всего для уничтожения внутренних данных, которые выделены в динамической памяти.

Конструктор должен начинаться с имени класса и у него не должно быть никакого типа перед названием. Конструктор обязательно должен иметь модификатор доступа \lstinline|public|. Как и для обычной функции, в конструктор можно передавать различные параметры.

Пример конструктора с параметрами:

\begin{lstlisting}
class Figure
{
private:
    double square;

public:
    Figure(double figureSquare) // constructor
    {
        square = figureSquare;
    }
};

int main()
{
    Figure circle(15.0); // circle.square = 15.0;
    ...

    return 0;
}
\end{lstlisting}

Также существует \emph{конструктор по умолчанию}. Таким конструктором является всякий конструктор, у которого нет параметров и вызывается он в том случае, если создается объект без параметров. Если он явно не указан, то он генерируется автоматически с пустым содержимым.

\begin{lstlisting}
class Figure
{
private:
    double square;

public:
    Figure() // default constructor
    {
        square = 0;
    }

    Figure(double figureSquare) // constructor
    {
        square = figureSquare;
    }
};

int main()
{
    Figure circle(15.0); // circle.square = 15.0;
    Figure triangle; // triangle.square = 0.0;
    ...

    return 0;
}
\end{lstlisting}

Деструктор объявляется практически также, как и конструктор, но к названию добавляется символ~\lstinline|~|:

\begin{lstlisting}
class Figure
{
public:
    ~Figure() // destructor
    {
        ...
    }
};
\end{lstlisting}

\subsubsection{Разделение класса на два файла. Отделения интерфейса от реализации}
В языке \texttt{C++} существует возможность разделить класс на \emph{интерфейс} и \emph{реализацию}. Интерфейс~---~описание свойств и методов класса без раскрытия деталей реализации. Реализация~---~это собственно программный код, которые реализует поставленные методы класса.

Интерфейс класса необходимо указывать в файле \lstinline|className.h|, где \emph{className}~---~название класса. В данном файле можно указать заголовочные файлы, которыми будет пользоваться класс; указать используемые пространства имен, а также указать сам класс с его свойствами и \emph{прототипами методов}, которые включают в себя имена методов, списки параметров методов и их возвращаемые значения.

Пример интерфейса класса \emph{Student}:

\emph{student.h}
\begin{lstlisting}
#ifndef STUDENT_H
#define STUDENT_H

#include<string>

using namespace std;

class Student
{
private:
    string studentName;
    string studentSurname;
    int rating;
    int ratingSize;

public:
    Student(string name, string surname);

    string getStudentName();
    void setStudentName(const string& name);

    string getStudentSurname();
    void setStudentSurname(const string& surname);

    void addMark(int mark);
    double getAverageRating();

    void printInformation();
};

#endif
\end{lstlisting}

В начале заголовочного файла ставится защита подключения~---~\emph{include guard}:

\begin{lstlisting}
#ifndef CLASSNAME_H
#define CLASSNAME_H
...
#endif
\end{lstlisting}

, где вместо \lstinline|CLASSNAME|~---~название класса. Данный механизм используется для того, чтобы в коде при многочисленном обращении к данному заголовочному файлу оставался один экзмепляр данного файла, в противном случае возникнет ошибка компилятора по поводу дублирования файлов. Внутри данной конструкции необходимо перечислить все то, что необходимо защитить от копирования, в том числе и интерфейс классов.

Обратите внимание, что в заголовочном классе нет реализации методов класса, вместо них указаны только их прототипы.

В данном примере рассматривается класс \lstinline|Student|. У него есть свойства \lstinline|studentName| (имя студента), \lstinline|studentSurname| (фамилия студента), \lstinline|rating| (сумма баллов), \lstinline|ratingSize| (количество введеных баллов). Для класса cуществует конструктор \lstinline|Student|, который принимает на вход имя и фамилию студента; \lstinline|get| и \lstinline|set| методы для свойств \lstinline|studentName| и \lstinline|studentSurname|; метод \lstinline|addMark|, который добавляет оценку к свойству \lstinline|rating|; метод \lstinline|getAverageRating|, который выводит средний балл студента и метод \lstinline|printInformation|, который выводит всю информацию о студенте.

Реализацию класса необходимо указывать в файле \lstinline|className.cpp|, где \emph{className}~---~название класса. В нем необходимо указать необходимые заголовочные файлы, пространства имен, а также \textbf{обязательно} указать заголовочный файл, в котором находится интерфейс данного класса.

Пример реализации класса \emph{Student}:

\emph{student.cpp}
\begin{lstlisting}
#include "student.h"
#include <iostream>

Student::Student(string name, string surname)
{
    studentName = name;
    studentSurname = surname;
    rating = 0;
    ratingSize = 0;
}

string Student::getStudentName()
{
    return studentName;
}

void Student::setStudentName(const string& name)
{
    studentName = name;
}

string Student::getStudentSurname()
{
    return studentSurname;
}

void Student::setStudentSurname(const string& surname)
{
    studentSurname = surname;
}

void Student::addMark(int mark)
{
    if (mark > 0)
    {
        rating += mark;
        ++ratingSize;
    }
}

double Student::getAverageRating()
{
    return (double)(rating) / ratingSize;
}

void Student::printInformation()
{
    cout << "Student " << studentSurname << " " << studentName << " has average rating: " << getAverageRating() << endl;
}
\end{lstlisting}

Каждое имя метода предваряется именем класса и \lstinline|::|, \emph{бинарной (двухместной) операцией разрешения области действия}. Таким образом каждая реализация методов <<привязывается>> к соответствующему методу к интерфейсу. Если не указывать \lstinline|::|, то связывания не произойдет!

Для того, чтобы использовать данный разделенный класс в своих программах, также необходимо указать заголовочный файл с интерфейсом класса:

\begin{lstlisting}
#include "student.h"

int main()
{
    Student firstStudent("Mark", "Ivanov");

    firstStudent.addMark(4);
    firstStudent.addMark(5);
    firstStudent.addMark(5);

    firstSrudent.printInformation();

    return 0;
}
\end{lstlisting}

\subsection{Наследование}
\subsubsection{Введение}
Наследование --- это способ повторного использования программного обеспечения, при котором новые \emph{производные} классы (наследники) создаются на базе уже существующих \emph{базовых} классов (родителей).

При создании новый класс является наследником свойств и методов ранее определенного базового класса. Создаваемый путем наследования класс является \emph{производным} (derived class), который в свою очередь может выступать в качестве \emph{базового} класса (based class) для создаваемых классов. Если имена методов производного и базового классов совпадают, то методы производного класса перегружают методы базового класса.

Наследование производится следующим образом:

\begin{lstlisting}
class DerivedClassName : typeInheritance BaseClassName
{
public:
    ...
private:
    ...
protected:
    ...
};
\end{lstlisting}

, где \lstinline|DerivedClassName|~---~производный класс, \lstinline|typeInheritance|~---~способ наследования, \lstinline|BaseClassName|~---~базовый класс.

Пример наследования:
\begin{lstlisting}
class Human
{
public:
    std::string name;
    unsigned int age;

    Human(std::string humanName, unsigned int humanAge)
    {
        name = humanName;
        age = humanAge;
    }
};

class Student : public Human
{
public:
    std::string university;

    Student(std::string humanName, unsigned int humanAge, std::string studentUniversity) : Human(humanName, humanAge)
    {
        university = studentUniversity;
    }
};

int main()
{
    Student student("Anton", 23, "NOVSU");
    std::cout << "Student " << student.name << ", age " << student.age << ", studies in " << student.university << std::endl;
    // Student Anton, age 23, studies in NOVSU
    return 0;
}
\end{lstlisting}

В данном примере рассмотрен класс \lstinline|Human|, у которого есть свойства \lstinline|name| (имя человека) и \lstinline|age| (возраст). При помощи конструктора можно инициализировать эти свойства. Класс \lstinline|Student| наследуется от класса \lstinline|Human|, в результате чего также владеет свойствами \lstinline|name| и \lstinline|age|, а также добавляет своё свойство \lstinline|university| (университет, где он учится). Каждый базовый класс, который участвует в наследовании, должен быть инициализирован во время инициализации своего потомка. Поэтому в конструктор класса \lstinline|Student| передаются параметры для класса \lstinline|Human|, и данные параметры используется в конструкторе класса \lstinline|Human|.

\subsubsection{Модификатор доступа protected}
Если в базовом классе свойства и методы имеют модификатор доступа \lstinline|private|, то производный класс не будет иметь доступ к ним при наследовании. Если свойства и методы будут иметь модификатор доступа \lstinline|public|, то доступ будут иметь абсолютно все, что не есть хорошо. При помощи модификатора \lstinline|protected| можно разрешить производному классу обращаться к свойствам и методам базового класса, но при этому доступ извне к ним не будет осуществляться!

Пример:

\begin{lstlisting}
class Rectangle
{
protected:
    double width, length;
public:
    Rectangle(double recWidth, double recLength)
    {
        width = recWidth;
        length = recLength;
    }

    double getSquare()
    {
        return width * length;
    }
};

class Cuboid : public Rectangle
{
private:
    double height;
public:
    Cuboid(double cubWidth, double cubLength, double cubHeight) : Rectangle(cubWidth, cubLength)
    {
        height = cubHeight;
    }

    double getVolume()
    {
        return width * length * height;
    }
};

int main()
{
    Rectangle rectangle(5.0, 3.0);
    Cuboid cuboid(10.0, 6.0, 2.0);

    std::cout << "Rectangle's square = " << rectangle.getSquare() << std::endl;
    std::cout << "Cuboid's bottom square = " << cuboid.getSquare() << ", cuboid's volume = " << cuboid.getVolume() << std::endl;

    // Rectangle's square = 15
    // Cuboid's bottom square = 60, cuboid's volume = 120

    return 0;
}
\end{lstlisting}

В данном примере рассматриваются классы \lstinline|Rectangle| (прямоугольник) и \lstinline|Cuboid| (прямоугольный параллелепипед). У класса \lstinline|Rectangle| есть свойства \lstinline|width| (ширина) и \lstinline|length| (длина), которые имею модификатор доступа \lstinline|protected|, и конструктор, в который передаются значения ширины и высоты, а также публичный метод \lstinline|getSquare|, который позволяет получить площадь прямоугольника. Класс \lstinline|Cuboid| наследуется от класса \lstinline|Rectangle|, имеет доступ к свойствами \lstinline|width| и \lstinline|length|, добавляет своё свойство \lstinline|height| (высота) и новый метод \lstinline|getVolume|, позволяющий получить объем. В методе \lstinline|getVolume| класса \lstinline|Cuboid| возвращает произведение длины, ширины и высоты, доступ к длине и ширине базового класса обеспечивается модификатором доступа \lstinline|protected|.

\subsubsection{Типы наследования}

При наследовании классов можно указать тип наследования, которые могут модифицировать существующие модификаторы доступа. Следующая табличка раскрывает взаимодействие типов наследования и модификаторов доступа:

\begin{table}[h]
    \centering
    \begin{tabular}{|>{\centering\arraybackslash}m{3cm}|>{\centering\arraybackslash}m{3cm}|>{\centering\arraybackslash}m{3cm}|>{\centering\arraybackslash}m{3cm}|}
        \hline
        & \multicolumn{3}{c|}{Исходный модификатор доступа} \\
        \hline
        & \lstinline|public| & \lstinline|private| & \lstinline|protected| \\ 
        \hline
        public-наследование & \cellcolor{green}\emph{public} & \cellcolor{red}\emph{private} & \cellcolor{blue}\emph{protected} \\
        \hline
        private-наследование & \cellcolor{red}\emph{private} & \cellcolor{red}\emph{private} & \cellcolor{red}\emph{private} \\
        \hline
        protected-наследование & \cellcolor{blue}\emph{protected} & \cellcolor{red}\emph{private} & \cellcolor{blue}\emph{protected} \\
        \hline
    \end{tabular}
\end{table}

\subsubsection{Множественное наследование}